\documentclass[12pt]{article}
\usepackage{geometry}
\usepackage{amsmath}
\usepackage{amsthm}
\usepackage{amssymb}
\usepackage{mathrsfs}
\usepackage{parskip}
\usepackage{enumerate}
\usepackage{stmaryrd}
\usepackage{listings}
\usepackage{fullpage}

\begin{document}

\title{CS 341 Notes}
\author{Matthew Visser}
\date{Oct 11, 2011}
\maketitle

\section{Greedy Algorithms}

Typically this is for optimization, minimization, maximization.

Greedy algorithms take what is currently the best answer and returns it.

\begin{itemize}
	\item Simple idea, but doesn't always work
		\begin{itemize}
			\item used to solve optimization problems that involve choosing
				objects, events, from a set
		\end{itemize}
	\item The strategy
		\begin{itemize}
			\item at each step, make the best next choice
			\item Never backtrack or change past choices
			\item never look ahead to see if this choice has negative
				consequences.
		\end{itemize}
	\item Depending on the problem, this may or may not give a good answer.
	\item The text treats greedy algorithms as a special case of dynamic
		programming.
\end{itemize}

\subsection{Example --- Make Change}

How do we give back change for payment minimizing the amount of coins we give
back. The coins are in denominations of:
\begin{itemize}
	\item 1c
	\item 5c
	\item 10c
	\item 25c
	\item 50c
	\item 100c
	\item 200c
\end{itemize}

Say we have to give out 350c in change. What is the minimum amount of change we
give back?

Find the biggest coin that fits, and take as many of it as we can. This gives us
6 coins that we give.

This works fine, but how can we prove that this will give us the right answer?

Consider if we had coins of 1c, 49c, 51c, 53c. The greedy algorithm we have
would give us 48 coins when the optimal solution is 2 coins. So this algorithm
doesn't always give a solution if the denominational system is different, and
doesn't always give a good solution for this.

\subsection{Properties of Greedy Algorithms}

\begin{itemize}
	\item Usually very simple and very fast. Considering out example it was
		$O(m)$ where $m$ is the number of coind denominations we have.
	\item Not really easy to prove it works all the time.
	\item At each stage we have a partial solution that leads us to the new
		solution.
\end{itemize}

\subsection{Example --- Scheduling Activities}

\textbf{Instance:} a set of $n$ activities, each with start times $s_i$ and
finishing time $f_i$.

\textbf{Problem}: find the subset of activities with the maximum number of
non-overlapping activities.

\begin{center}
\emph{*** See slide ``Scheduling Activities'' ***}
\end{center}

We say activities $i$ and $j$ are \emph{compatible} if the do \textbf{not}
overlap, \textit{i.e.}, $s_i \ge f_j$ or $s_j \ge f_i$.

Chose the activity with the earliest finishing time, delete it, and those that
are incompatible with the chosen activity, then repeat.

\begin{itemize}
	\item Sort activities by finishing time
	\item choose first activity
	\item repeatedly choose next activity that is compatible with all previous
		choices
\end{itemize}

\textbf{Notation}:

\begin{itemize}
	\item Let $S = \{a_1, a_2,\dots,a_n\}$ be the activities
	\item Assume each one is represented by a half-open interval, \textit{i.e.},
		$a_i = [s(a_i),f(a_i))$.
	\item Sort the list of activities such that $f(a_1 \le f(a_w\le \dots \le
		f(a_n)$.
	\item So compatibility of $a_i$ and $a_k$ implies that $s(a_k) \ge f(a_i)$
		or $s(a_i) \ge f(a_k)$.
\end{itemize}

\textbf{Proof}:

Prove greedy selection will give an optimal solution for the activity selection
problem.

\begin{equation}
	G = \{g_1,g_2,g_3,\dots g_m\}
\end{equation}

\begin{equation}
	H = \{h_1,h_2,\dots,h_k,h_{k+1},\dots,h_m\}
\end{equation}

We will inductively show that we can replace all choices with the greedy choice
and maintain optimality.

We know that
\begin{equation}
	f(g_1) \le f(h_1)
	\label{ineq}
\end{equation}
The finishing value of $g_1$ is smaller or equal to the finishing value of
$h_1$.

Now we change $h_1$ to $g_1$. Does this destroy the ordering for $H$? We know
$h_1$ and $h_2$ are compatible, $s(h_2) \ge f(h_1)$. We take equation
\eqref{ineq}, and we know that $g_1$ and $g_2$ are compatible.

So we know that we have roughly equivalent solutions to a certain point. We have
\begin{equation}
	H = \{g_1,g_2,\dots,g_k,h_{k+1},\dots,h_m\}
	\label{newH}
\end{equation}
We say
\begin{equation}
	H^* = \{h_{k+1},\dots,h_m\}
	\label{starH}
\end{equation}

$H^*$ is compatible with any events before it. We can claim $H^*$ is the optimal
solution to a subproblem of the problem before it. Then we can replace $H$ with
$G$ and this will still be an optimal solution.

\subsection{Scheduling Activities into Rooms}

See notes.

\subsection{Fractional Knapsack}

A thief breaks into a warehouse full of bulk materials. Each type of material is
conveniently labelled with the price per pound.

\end{document}
% vim: tw=80
