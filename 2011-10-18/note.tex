\documentclass[12pt]{article}
\usepackage{geometry}
\usepackage{amsmath}
\usepackage{amsthm}
\usepackage{amssymb}
\usepackage{mathrsfs}
\usepackage{parskip}
\usepackage{enumerate}
\usepackage{stmaryrd}
\usepackage{listings}
\usepackage{fullpage}

\begin{document}

\title{CS 341 Notes}
\author{Matthew Visser}
\date{Oct 18, 2011}
\maketitle

\numberwithin{equation}{subsection}

\section{Dynamic Programming}

The problem we want to solve must have some sort of recursive nature, or
subproblems.

Dynamic programming sort of implies that we have two passes. One to calculate
the optimal value, and one to calculate the solution for that value.

\subsection{Terminology}

See slides

\subsection{Notes on Dynamic Programming}

\begin{itemize}
	\item Subproblems are the core of dynamic programming. We want an array of
		solutions to subproblems.
	\item Three steps
		\begin{enumerate}
			\item Describe subproblems
			\item Determine how an array will be used to evaluate scores
				\begin{itemize}
					\item What are the dimensions, how do we store, what is the
						base case, where do we find the optimal score?
				\end{itemize}
			\item Recovery of the solution --- Where do we trace back to find
				the solution.
		\end{enumerate}
\end{itemize}

\subsection{Example --- Longest Common Subsequence}

Some notation:

\begin{itemize}
	\item A \emph{subsequence} of a string $X$ is a string that can be obtained
		by deleting some of the characters from $X$. Characters maintain their
		original order.
	\item A \emph{prefix} of $X$ is a number of characters starting from the
		front of a string. Denote the $i$\textsuperscript{th} prefix by $X[i]$.
\end{itemize}

We are given $X = x_1,\dots,x_m$ and $Y = y_1,\dots,y_n$.

Is the LCS unique? No.

We take a look at the arrays, with indexes $i$ and $j$. If $x_{i} = y_{j}$,
store it at $z_k$.

We have a recurrance for LCS:
\begin{equation}
	C[i,j] = \begin{cases}
		C[i-1,j=1]+1 & X[i] = Y[j]\\
		\max \{C[i-1,j],C[i,j-1]\} & X[i]\neq Y[j]\\
	\end{cases}
\end{equation}

\subsection{Example --- The Integer Knapsack Problem\slash 0-1 Knapsack problem}

We previously saw the fractional knapsack problem. This is similar, but we can't
put fractions of objects into the knapsack.

We can see the greedy algorithm won't work.  We try dynamic programming.

Imagine all objects numbered $1\dots n$. We specify score by letting $V[i,j]$ be
the maximum value of the objects, selected from the first $i$ objects, that can
fit into a knapsack with upper weight limit $j$. The final optimal solution will
be in $V[n,W]$.

\textbf{Key Observation}: We either use object $i$ or we do not.
\begin{itemize}
	\item Suppose $i$ is not in the knapsack. Then there is no difference
		between $V[i-1,j]$ and $V[i,j]$.
	\item Suppose object $i$ is in the knapsack.
		\begin{itemize}
			\item Claim: In this case $V[i,j] = V[i=1,j-w_i] + v_i$.
		\end{itemize}
\end{itemize}

We have a recurrence
\begin{equation}
	V[i,j] = \max \{V[i-1,j], v_i + V[i-1,j-w_i]\}
\end{equation}

\end{document}
% vim: tw=80
