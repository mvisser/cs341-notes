\documentclass[12pt]{article}
\usepackage{geometry}
\usepackage{amsmath}
\usepackage{amsthm}
\usepackage{amssymb}
\usepackage{mathrsfs}
\usepackage{parskip}
\usepackage{enumerate}
\usepackage{stmaryrd}
\usepackage{listings}
\usepackage{fullpage}
\usepackage{qtree}
\usepackage{algorithmic}
\usepackage{algorithm}

\begin{document}

\title{CS 341 Notes}
\author{Matthew Visser}
\date{Nov  8, 2011}
\maketitle

\newtheorem{thm}{Theorem}

\section{Graph Algorithms}
\subsection{Articulations}

Consider the spanning tree:
\Tree [ .{$r$}  \qroof{children of 1}.1    \qroof{ children of 2 }.2   ]

Then $r,1,2$ are vertices that when removed, the graph is diconnected if some
conditions talked about last class are met.

\subsection{Minimum Spanning Trees}

\begin{description}
	\item[Invariant:]  $T$ is a subgraph of some Minimum Spanning Tree (MST).
	\item[Safe:] An edge $e$ is safe for $T$ if adding $e$ to $T$ maintains that
		it is a subgraph of a MST.
\end{description}

$MST(G,w),\ G = (V,E),\ w: E \to \mathbb{R}$.

$T$ = an empty graph.

\begin{verbatim}
while T is not a spanning tree do
find a safe edge e
add edge to T
\end{verbatim}

\begin{tabular}{p{0.5\textwidth}|p{0.5\textwidth}}
	Kruskal & Prim\\
	\hline
	\begin{itemize}
		\item grow a forest
		\item Add edges with the least weight --- as long as they don't violate
			the spanning tree condition.
		\item Cost is $O(e)$
	\end{itemize}
	& \begin{itemize}
		\item grow a single tree
	\end{itemize}\\
\end{tabular}

\begin{algorithm}
	\label{alg:kruskal}
	\caption{Kruskal}
	\begin{algorithmic}[1]
		\STATE{$T \gets \emptyset$}
		\FOR{$i \gets 1 \to m$}
		\STATE{add edge if it doesn't violate spanning tree condition}
		\ENDFOR
	\end{algorithmic}
\end{algorithm}

\begin{thm}
	Kruskal's algorithm returns an MST.
	\begin{proof}
		Consider $e_1 \le e_2 \le \dots \le e_{k-1} \le e_k$. We will prove that
		this is true by induction on $k$.

		\textbf{Base Case:} $k=0$. Trivial

		\textbf{Inductive Step:}

		Consider 
		\begin{equation}
			T^*: e_1 \le e_2 \le e_3 \le \dots \le e_{k-1} \le e^*_k \le
			e_{k+1}^* \le \dots \le e_{n-1}^*
			\label{eq:pkrus1}
		\end{equation}

		Let $T' = T^* + e_k - e_j^*$. To do this, we have to subtract another edge that
		exists on a loop that may exist with $e_k$ in it. We're swapping
		the edge $e_k$ with $e_j^*$, and we also know that $e_k \le e_j^*$ since
		neither are in the spanning tree yet and both can be added. $\therefore
		T' \le T^*$ and the algorithm returns an MST.
	\end{proof}
\end{thm}




\end{document}
% vim: tw=80
