\documentclass[12pt]{article}
\usepackage{geometry}
\usepackage{amsmath}
\usepackage{amsthm}
\usepackage{amssymb}
\usepackage{mathrsfs}
\usepackage{parskip}
\usepackage{enumerate}
\usepackage{stmaryrd}
\usepackage{listings}
\usepackage{fullpage}
\usepackage{hyperref}

\begin{document}

\title{CS 341 Notes}
\author{Matthew Visser}
\date{Sep 13, 2011}
\maketitle

\section*{Intro}

\begin{description}
    \item[Web page:] \url{http://www.student.cs.uwaterloo.ca/~cs341}
    \item[Textbook:] \textit{Introduction to Algorithms} (3rd. ed.) Cormen,
        Leiserson, Rivest, and Stein
    \item[Newsgroup:] uw.cs.cs341
\end{description}

\section*{Algorithms in Practice}

\begin{description}
    \item[Algorithm:] A way of solving a problem. There are many algorithms for
        a specific problem. Origin of the word is from ``al-Khowarizmi'', the
        name of the mathemetician who came up with the idea in a book. In latin,
        the title was writen as ``Algoritmi de numero Indorum'' or
        ``al-Khorwarizmi's Book on Indian Numbers''.
    \item[Program:] An implementation of an algorithm. There are many
        implementations of an algorithm.
\end{description}


What makes a good algorithm?

\begin{itemize}
    \item It is \textbf{correct}.
        \begin{itemize}
            \item Consider creating an RSA key, the number has to be prime, but
                you have to know that it is.
            \item Must be able to prove it's correct.
        \end{itemize}
    \item It is \textbf{efficient}.
        \begin{itemize}
            \item want this done \textit{today} (or soon after).
        \end{itemize}
\end{itemize}


\end{document}
% vim: tw=80
