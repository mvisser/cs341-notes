\documentclass[12pt]{article}
\usepackage{geometry}
\usepackage{amsmath}
\usepackage{amsthm}
\usepackage{amssymb}
\usepackage{mathrsfs}
\usepackage{parskip}
\usepackage{enumerate}
\usepackage{stmaryrd}
\usepackage{listings}
\usepackage{fullpage}
\usepackage{qtree}

\begin{document}

\title{CS 341 Notes}
\author{Matthew Visser}
\date{Sep 27, 2011}
\maketitle

The runtime for fast integer multiplication can be expressed with 
\begin{eqnarray}
    T(1) &=&  1\\
    T(n) &=&  3T\left(\frac{n}{2}\right) + \Theta(n)
\end{eqnarray}

There are three methods for solving recurrences:
\begin{itemize}
    \item Recursion Tress
    \item Master Theorem
    \item Guess and Check
\end{itemize}

\section{Recursion Trees}

We will analyze mergesort, on page 1 of recursion trees slides.

We have the reccurence
\begin{eqnarray}
    T(1) &=&  0\\
    T(n) &=&  T\left( \lfloor \frac{n}{2}\rfloor \right) + T\left( \lceil
    \frac{n}{2}\rceil \right) + \Theta(n)
\end{eqnarray}

\begin{eqnarray}
    T(1) &=&  0\\
    T(n) &\leq&  T\left( \lfloor \frac{n}{2}\rfloor \right) + T\left( \lceil
    \frac{n}{2}\rceil \right) + an
\end{eqnarray}

\Tree
[
.{$T(n)$}
[
.{$T(\lfloor \frac{n}{2}\rfloor$}
{$T(\lfloor \frac{\lfloor\frac{n}{2}\rfloor}{2}\rfloor$}
{$T(\lfloor \frac{\lceil\frac{n}{2}\rceil}{2}\rfloor$}
]
{$T(\lceil \frac{n}{2}\rceil$}
]

\Tree
[
.{$an$}
[
.{$a\lfloor \frac{n}{2}\rfloor$}
{$a\lfloor \frac{\lfloor\frac{n}{2}\rfloor}{2}\rfloor$}
{$a\lfloor \frac{\lceil\frac{n}{2}\rceil}{2}\rfloor$}
]
{$a\lceil \frac{n}{2}\rceil$}
]

\textbf{Cost per level}:

\begin{enumerate}
    \item $an$
    \item $an + \lceil \frac{n}{2} \rceil + \lfloor \frac{n}{2}\rfloor$
\end{enumerate}

This is getting tedious with floors and ceilings, so ignore them and use
$T(n) \leq 2T(\frac{n}{2} + an$.

Try again.
\Tree
[
.{$an$}
[
.{$a \frac{n}{2}$}
{$a \frac{n}{4}$}
{$a \frac{n}{4}$}
]
{$a \frac{n}{2}$}
]

Now we see for every level we have $an$ as a cost. Each level $x$ has $2^x$
nodes. We know we have approximately $\log(n)$ levels, so we end up with
$an\cdot \log(n)$ as a cost. So
\begin{eqnarray}
    T &=&  \sum_{i=0}^{k-1}an \\
    T &=&  an\cdot \log(n) \\
    T &=& O(n \log n)
\end{eqnarray}

Note that we made a lot of assumptions about the above equation --- We assumed
that $n$ was a power of 2, and that we could get rid of the floor and ceiling
functions.

\subsection{Examples}

\begin{enumerate}
    \item Karatsuba integer multiplication

        The recursion is:
        \begin{equation}
            T(n) = \begin{cases}
                1 & n = 1\\
                3T\left( \frac{n}{2} \right) + \Theta(n) & n > 1
            \end{cases}
        \end{equation}

        \Tree
        [
        .{$an$}
        [
        .{$a\frac{n}{2}$}
        [
        .{$a\frac{n}{4}$}
        [
        .{$a\frac{n}{2^3}$}
        ]
        [
        .{$a\frac{n}{2^3}$}
        ]
        [
        .{$a\frac{n}{2^3}$}
        ]
        ]
        [
        .{$a\frac{n}{2^2}$}
        {\dots}
        ]
        [
        .{$a\frac{n}{4}$}
        {\dots}
        ]
        ]
        [
        .{$a\frac{n}{2}$}
        {\dots}
        ]
        [
        .{$a\frac{n}{2}$}
        {\dots}
        ]
        ]

        At every level we have $\frac{an\cdot 3^k}{2^k}$ and at the last level
        we have $3^k$, one for each node.

        So we have 
        \begin{eqnarray}
            T(n) &=&  3^k + \sum_{i=0}^{\log_2(n)-1}
            an\left(\frac{3}{2}\right)^i\\
            T(n) &=&  3^k + an\frac{\left(\frac{3}{2}\right)^{\log
            n}-1}{\left(\frac{3}{2}\right) - 1}\\
            T(n) &\leq&  3^k + an^{\log(3)}\\
            T(n) &\leq&  (a+1)n^{\log(3)}\\
            T(n) &=&  O(n^{\log3})
        \end{eqnarray}
    \item Strassen
\end{enumerate}

\section{Master Theorem}

The master theorem gives the solution of recurrences of the form $T(n) = aT(n/b)
+ f(n)$.

There are three cases for the theorem.  It is applicable even when the
recurrences involve the standard floors and ceilings in the partition, so long
as the sloppy form looks like the above equation.

\textbf{Theorem}: Let $a \geq 1, b > 1$ be constants. Let
\begin{equation}
    T(n) = aT\left(\frac{n}{b}\right) + f(n)
    \label{MasterCond}
\end{equation}
where $n/b$ denotes either $\lceil n/b\rceil$ or  $\lfloor n/b \rfloor$. Then
\begin{equation}
    T(n) \in \begin{cases}
        \Theta(n^{\log_b(a)}) & f(n) \in O(n^{\log_b(a - \varepsilon)})\\
        \Theta(n^{\log_b(a)}\lg(n) & f(n)) \in \Theta(n^{\log_b(a)})\\
        \Theta(f(n)) &f(n)\in \Omega(n^{\log_b(a + \varepsilon)}) \text{ and }
        af\left( \frac{n}{b} \right) \leq cf(n)\\
    \end{cases}
    \label{MasterTheorem}
\end{equation}

\subsection{Example}

We look at merge sort again.

We have
\begin{equation}
    T(n) = 2T\left( \frac{n}{2} \right) + an
\end{equation}

$n^{\log_b a} = n^1$. Then $f(n) \in \Theta(n) = \Theta(n^{\log_b a}$

Then by master theorem we get $T(n) \in \Theta( n \log n)$

\subsection{Another Example}

We have
\begin{equation}
    T(n) = 3T\left( \frac{n}{2} \right) + an
\end{equation}

$n^{\log_b a} = n^{\log_2 3} \approx n^{1.58}$.

$f(n) = an \in O(n^{1.5})$ or $f(n) \in O(n^{\log_2 3 - \varepsilon})$ which
follows the first case of the master theorem.

THen by master theorem $T(n) \in \Theta(n^{\log 3})$

\subsection{An Example That We Can't Do\dots}

Consider $f(n) \in \Theta(n^x \log n)$. Cases 1 and 2 don't apply. Case 3
doesn't apply either because $n^x \log n$ isn't of the form $\Omega(n^{\log_b a
+ \varepsilon}) = \Omega(n^{x + \varepsilon})$ no matter how small $\varepsilon
> 0$ is.



\end{document}
% vim: tw=80
