\documentclass[12pt]{article}
\usepackage{geometry}
\usepackage{amsmath}
\usepackage{amsthm}
\usepackage{amssymb}
\usepackage{mathrsfs}
\usepackage{parskip}
\usepackage{enumerate}
\usepackage{stmaryrd}
\usepackage{listings}
\usepackage{fullpage}

\begin{document}

\title{CS 341 Notes}
\author{Matthew Visser}
\date{Nov 10, 2011}
\maketitle

\section{Graph Algorithms}

\subsection{Prim's Algorithm}

\begin{enumerate}
	\item Give each node an infinite cost except for our start node, which has 0.
	\item Grab the vertext $V$ with the smallest cost.
	\item For this node, update all its neighbors with $\min\{C,C(E)\}$ smallest
		cost. The minimum of the cost at the node currently and the current node
		plus the cost for the edge. Follow the edge with the smallest cost. if
		that's the minimum.
	\item Repeate at 2 until all nodes are visited.
\end{enumerate}

\subsection{Minimal Path}

We want to start from a node and find the mimimum cost for a path from node $A$
to node $B$.

\textbf{Dykstra's Algorithm}

\begin{enumerate}
	\item Start at node $A$. Its cost is 0.
	\item For each neighbor of the current node $N$, if the cost to get to $N$
		plus the cost of the edge to the neighbor is less than the cost of the
		neighbor, update to the new cost and mark it as reachable from $N$.
	\item If the cost of the neighbor plus the cost of the edge is less, mark
		this node as that cost and set it to reachable from the neighbor.
	\item Repeat for vertex with lowest cost.
\end{enumerate}

Can we use this for negatives? No, because a negative edge could be taken
infinitely many times and always reduce cost.

\subsection{Floyd-Warshall}

We can use dynamic programming to find this. Use a recursive definition with a
base case of \texttt{cost}$(u,v,0)$. See notes for a better description, but the
general idea is just recurse on path length. The running time is cubic. The
first memory space we looked at was $O(n^3)$, but we can get away with $O(n^2)$.

\end{document}
% vim: tw=80
