\documentclass[12pt]{article}
\usepackage{geometry}
\usepackage{amsmath}
\usepackage{amsthm}
\usepackage{amssymb}
\usepackage{mathrsfs}
\usepackage{parskip}
\usepackage{enumerate}
\usepackage{stmaryrd}
\usepackage{listings}
\usepackage{fullpage}

\begin{document}

\title{CS 341 Notes}
\author{Matthew Visser}
\date{Sep 22, 2011}
\maketitle

\section{Divide and Conquer}

In Latin, \textit{divide et impera}.

These algorithms are typically recursive.

Three components
\begin{enumerate}
    \item Divide
        \begin{itemize}
            \item Would like these to be equally sized.
            \item Minimize \# of subproblems we could have.
            \item Sub-problems should be ``easy'' to compute.
        \end{itemize}
    \item Conquer
        \begin{itemize}
            \item Subproblems $\to$ independent.
        \end{itemize}
    \item Combine
        \begin{itemize}
            \item Shouldn't be ``too'' hard. If it is, it can increase time
                complexity.
        \end{itemize}
\end{enumerate}

Consider $n!$:
\[
n! = \begin{cases}
    1 & n=1,0\\
    n \times (n-1)! & n>1
\end{cases}
\]

We're breaking the problem down, but have very small steps.  Consider also
quicksort. It can have a bad worst case if we choose a bad pivot.  Mergesort
does better --- it always splits the problem into two pieces.

\subsection{Example --- Merge Sort}

Consider mergesort:
\[
T(n) = \begin{cases}
    0 & n=1\\
    T(\lfloor \frac{n}{2} \rfloor + T(\lceil \frac{n}{2} \rceil) + T(merge) &
    n>1
\end{cases}
\]

Assume $ T(merge) \in \Theta(n)$.
\[
T(n) = \begin{cases}
    0 & n=1\\
    T(\lfloor \frac{n}{2} \rfloor + T(\lceil \frac{n}{2} \rceil) + \Theta(n) &
    n>1
\end{cases}
\]

Assume that $n=2^k$.
\begin{eqnarray*}
T\left(n\right) &=& 2T\left(\frac{n}{2}\right) + \Theta\left(n\right) \\
&\leq& 2T\left(\frac{n}{2}\right) + c\cdot n \\
&\leq& 2\cdot 2 T\left(\frac{n}{4}\right) + c\frac{n}{2} + cn \\
&\leq& 4\left(wT\left(\frac{n}{8}\right) + c\frac{n}{4}\right) + 2cn \\
&\leq& 2^3 T\left(\frac{n}{2^3}\right) + 3cn\\
&\leq& 2^k T(1) + k\cdot c \cdot n\\
&=& n\cdot T(1) + c \cdot n \cdot \log\left(n\right)
\end{eqnarray*}

This shows us that merge sort is $T(n) \in O(n \log n)$. Notice that as long as
$T(1) \in O(\log n)$, this still holds.

We can prove that this works for any number by using induction.
\textit{See Slides}.

\end{document}
% vim: tw=80
