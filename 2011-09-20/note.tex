\documentclass[12pt]{article}
\usepackage{geometry}
\usepackage{amsmath}
\usepackage{amsthm}
\usepackage{amssymb}
\usepackage{mathrsfs}
\usepackage{parskip}
\usepackage{enumerate}
\usepackage{stmaryrd}
\usepackage{listings}
\usepackage{fullpage}

\begin{document}

\title{CS 341 Notes}
\author{Matthew Visser}
\date{Sep 20, 2011}
\maketitle

\section{Runtime Analysis}

$f(n) \in O(g(n) \text{ iff } \exists c>0, n_0>0 \text{ such that } \forall n >
n_0 \quad 0 \leq f(n) \leq c\cdot g(n)$.

$f(n) \in o(g(n) \text{ iff } \forall c>0 \exists n_0>0 \text{ such that }
\forall n > n_0 \quad 0 \leq f(n) \leq c\cdot g(n)$.

Easy way to show $f(n) \in o(g(n)$ is:
\begin{equation}
    \lim_{n \to \infty} \frac{f(n)}{g(n)} = 0
\end{equation}

This is very nice if you use L'Hopital's rule.  This is not a good proof by
definition, but is a good way to show it's true.

\subsection{Example proving $O(n)$}

Show 
\begin{equation}
    10n^2 + 100n - 3 \in O(n^2)
\end{equation}

Show $\exists c,n_0$.

\begin{eqnarray}
    10n^2 + 100n-3 &\leq& cn^2\\
    100n - 3 &\leq& (c-10)n^2\\
    \underbrace{ 1-\frac{3}{100n}}_{<1} &\leq& \left( \frac{c-10}{100} \right)n\\
    \frac{c-10}{100}&=& 1\\
    c&=&110
\end{eqnarray}

\subsection{Another example for a sum}

Show
\begin{equation}
    \sum_{i=1}^{n}\log(i) \in \theta\left( n\log(n) \right)
\end{equation}

\begin{eqnarray}
    \sum_{i=1}^{n}\log(i) &\leq& \sum_{i=1}^{n}\log n = n \log n\\
    \sum_{i=1}^{n}\log(i) &\geq& \sum_{i=\lfloor\frac{n}{2}\rfloor}^n \log i\\
\end{eqnarray}

\subsection{O-notation Subtleties}

See slides page 10.

\section{Divide and Conquer}

Example: Mergesort.

\subsection{Example: Multiplying arbitrarily large numbers}

Have two numbers of size $n$. Then the output size is $2n$. The complexity of
multiplying two numbers is $\Theta(n^2)$. Is there a lower bound for multiplying
two integers? Yes, must be $\Omega(n)$ since we have to iterate every digit of
both numbers. But can we do better? Let's see.

Multiply
\begin{tabular}{r|l}
    $A$ & $A_12^{n} + \dots + A_0$\\
    $\times B$ & $B_12^{n} + \dots + B_0$\\
    \hline
\end{tabular}


\begin{equation}
    AB = \underbrace{A_1B_1}_{T(\frac{n}{2}}\underbrace{2^n}_{\Theta(n)} + (A_1B_0 + A_0B_1)2^{\frac{n}{2}} + \dots + A_0B_0
\end{equation}

Where $T(n)$ is matrix addition, $\Theta(n)$ is the exponentiation of 2. So the
operation is $4T\left(\frac{n}{2}\right) + \Theta(n)$ where
\begin{equation}
    T(n) = \begin{cases}
        \Theta(1),\quad n=1\\
        4T\left( \frac{n}{2} \right) + \Theta(n)
    \end{cases} = \Theta(n^2)
\end{equation}

\begin{equation}
    T(n) = \begin{cases}
        \Theta(1),\quad n=1\\
        2T\left( \frac{n}{2} \right) + \Theta(n)
    \end{cases} = \Theta(n \log n)
\end{equation}

\begin{equation}
    \underbrace{(A_1 + A_0)(B_1 + B_0)}_{T\left( \frac{n}{2} \right) +
    \Theta(n)} =\underbrace{ A_1 B_1}_{T\left( \frac{n}{2} \right)} + A_1 B_0 + A_0 B_1 + A_0 B_0
\end{equation}

So compute first and last terms, then subtract from multiplication.

\subsection{Example: Matrix Multiplication}

\begin{equation}
    AB = 
    \left[
    \begin{matrix}
        A_{11} & A_{12}\\
        A_{21} & A_{22}
    \end{matrix}
    \right]
    \cdot
    \left[
    \begin{matrix}
        B_{11} & B_{12}\\
        B_{21} & B_{22}
    \end{matrix}
    \right]
    =
    \left[
    \begin{matrix}
        [A_{11}B_{11} + A_{12}B_{21}]  & [\dots]\\
        [A_{21}B_{11} + A_{22}B_{21} & [\dots]
    \end{matrix}
    \right]
\end{equation}


\end{document}
% vim: tw=80
