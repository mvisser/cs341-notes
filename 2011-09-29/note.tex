\documentclass[12pt]{article}
\usepackage{geometry}
\usepackage{amsmath}
\usepackage{amsthm}
\usepackage{amssymb}
\usepackage{mathrsfs}
\usepackage{parskip}
\usepackage{enumerate}
\usepackage{stmaryrd}
\usepackage{listings}
\usepackage{fullpage}
\usepackage{qtree}

\begin{document}

\title{CS 341 Notes}
\author{Matthew Visser}
\date{Sep 29, 2011}
\maketitle

\section{Master Theorem}


\subsection{Proof of the Master Theorem}
On $n$ input.
\Tree
[
.{$f(n)$}
[
.{$f(\frac{n}{b})$}
{$f(\frac{n}{b^2})$}
{\dots}
{\dots}
{\dots}
]
{\dots}
{\dots}
[
.{$f(\frac{n}{b})$}
{\dots}
]
]

First level $a$ nodes, second $a^2$. So for the $k^\text{th}$ level we have
$a^kf(\frac{n}{b^k})$. We get to the base level when $n=b^k \implies k =
\log_b(n)$.

Our base case has: $a^k = a^{\log_b(n)} = n^{\log_b(a)}$

So assuming $n=b^\ell,\ \ell \in \mathbb{N}$.
\[
T(n) = T(1) n^{\log_b(a)} + \sum_{j=0}^{\log_b(a)-1} a^jf\left(\frac{n}{b^j}\right)
\]

There are a number of cases:
\begin{enumerate}
    \item \[
        f \in O(n^{\log_b(a) - \varepsilon}) \implies f(n) \leq
        cn^{\log_b(a)-\varepsilon}\]

        Then we have
        \begin{eqnarray*}
            &&\sum_{i=1}^{k-1}a^i c\cdot\left( \frac{n}{b^i} \right)^{\log_b(a) -
            \varepsilon}\\
            &=& cn^{\log_b(a)-\varepsilon} \sum \frac{a^i}{\left( b^i
            \right)^{\log_b(a)-\varepsilon}}\\
            &=& cn^{\log_b(a)-\varepsilon} \sum \frac{a^i}{
            \left(b^{\log_b(a)-\varepsilon}\right)^i}\\
            &=& cn^{\log_b(a)-\varepsilon} \sum \frac{a^i}{a^i}b^{\varepsilon i}\\
            &=& cn^{\log_b(a)-\varepsilon} \sum \left(b^\varepsilon\right)^i\\
            &=& cn^{\log_b(a)-\varepsilon} \frac{(b^\varepsilon)^k -
            1}{(b^\varepsilon)-1}\\
        \end{eqnarray*}

        We know $k = \log_b(n)$ so we get
        \begin{eqnarray*}
            && cn^{\log_b(a)-\varepsilon} \frac{(b^\varepsilon)^k -
            1}{(b^\varepsilon)-1}\\
            &=& cn^{\log_b(a)-\varepsilon}n^{-\varepsilon} \frac{n^\varepsilon -
            1}{b^\varepsilon-1}
        \end{eqnarray*}
        Since $b^\varepsilon-1$ is just a constant, we get
        \begin{eqnarray*}
            && cn^{\log_b(a)-\varepsilon}n^{-\varepsilon} \frac{n^\varepsilon -
            1}{b^\varepsilon-1}\\
            &=& \frac{c}{b^\varepsilon - 1}n^{\log_b(a)-\varepsilon}n^{-\varepsilon}
            (n^\varepsilon - 1)\\
            &=& dn^{\log_b(a)-\varepsilon}n^{-\varepsilon}
            (n^\varepsilon - 1)\\
            &\in& O(n^{\log_b(a)})
        \end{eqnarray*}
        Where $d$ is a constant.

    \item We have $f(n) \in \theta(n^{\log_b(a)}$:
        \begin{eqnarray*}
            &&\sum_{i=1}^{k-1} a^i c\left( \frac{n}{b^i} \right)^{\log_b(a)}\\
            &=&cn^{\log_b(a)}\sum_{i=1}^{k-1} \frac{a^i}{a^i}\\
            &=&ckn^{\log_b(a)}\\
            &=&c\log_b(n)n^{\log_b(a)}\\
            &\in&\Theta(\lg(n)n^{\log_b(a)})
        \end{eqnarray*}

    \item We have $f(n) \in \Omega(n^{\log_b(a)+\varepsilon}$:

        \[
        af\left( \frac{n}{b} \right)\leq cf(n),\ 0 < c < 1\\
        \]
        \begin{eqnarray*}
            &&\sum_{i=0}^{k-1}a^if\left( \frac{n}{b^i} \right)\\
            &=&f(n)+\sum_{i=1}^{k-1}a^if\left( \frac{n}{b^i} \right)\\ 
        \end{eqnarray*}
        This must be $\Omega(f(n))$ since we have a first of term of $f(n)$. We
        must now show that it is $O(f(n)$ to prove that the recursion is
        $\Theta(f(n))$.
        \begin{eqnarray*}
             \sum_{i=0}^{k-1}a^if\left( \frac{n}{b^i} \right) &=&  f(n) +
            af\left( \frac{n}{b} \right) + a^2f\left( \frac{n}{b^2} \right) +
            \dots\\
            &\leq& f(n) + cf(n) + acf\left( \frac{n}{b} \right) + \dots\\
            &\leq& f(n) + cf(n) + c^2f(n) + \dots
        \end{eqnarray*}
        This is because we know
        \[
        a^if\left( \frac{n}{b^i} \right) \leq c^if(n)
        \]
        So we get:
        \begin{eqnarray*}
            f(n)\sum_{i=0}^{k-1}c^i &=& f(n) \left( \frac{1-c^k}{1-c} \right)\\
            &=& f(n) \left( \frac{1-c^{\log_b(n)}}{1-c} \right)\\
            &=& f(n) \left( \frac{1-c^{\log_b(n)}}{1-c} \right)\\
        \end{eqnarray*}
        \begin{eqnarray*}
            f(n)\sum_{i=0}^{k-1}c^i &\leq& f(n)\sum_{i=0}^{\infty} c^i\\
            &=& f(n)\frac{1}{1-c}\\
            &\in&O(f(n))
        \end{eqnarray*}
        This is as required, so we now have $T(n) \in \Theta(f(n))$.
\end{enumerate}

This proof is also available in the slides and the textbook. 

\subsection{Using the Master Theorem}

\begin{itemize}
    \item State recurrences in precise form and sloppy form
    \item If the theorem applies, use it on sloppy form, then the solution will
        apply to the precise case
    \item If the theorem does not apply, use recursion trees, then check using
        induction.
\end{itemize}

\section{Guessing and Checking}

\textbf{Example}: $T(n) = T(n-1) + 1$. The total cost is $n$. We must prove this
(using induction).

\textbf{Base Case}: Assuming $T(0) = 0$ so the base case is trivially true.

\textbf{Inductive Step}:
Assume true for $n-1$. Then we have $T(n) = T(n-1) + 1 = n-1 + 1 = n$ as
required.

We may want to use guess and check if we already know the answer and the master
theorem may or not apply.

\textbf{Example}: Merge sort split into 3. We can guess that we have a running
time of $T(n) = n \log_3(n)$.

By induction, we have\dots

\textbf{Base Case}: $T(1) = 0$

\textbf{Inductive Step}:
We know that $T(n) = 3T\left(\frac{n}{3}\right) + n$. Assume this holds for
up to and including $n-1$.

\begin{eqnarray*}
    T(n) &=&  3(n/3 \cdot \log_3(n/3)\\
    &=& n \log_3(n/3) + n\\
    \dots\\
\end{eqnarray*}

Sometimes when we guess and check, we may need to add lower order terms to our
bounds.

\end{document}
% vim: tw=80
