\documentclass[12pt]{article}
\usepackage{geometry}
\usepackage{amsmath}
\usepackage{amsthm}
\usepackage{amssymb}
\usepackage{mathrsfs}
\usepackage{parskip}
\usepackage{enumerate}
\usepackage{stmaryrd}
\usepackage{listings}
\usepackage{fullpage}

\begin{document}

\title{CS 341 Notes}
\author{Matthew Visser}
\date{Nov 22, 2011}
\maketitle

\section{$P = NP$}

A decision problem $Q$ is in $P$ if there exists a polynomial time algorithm $A$
that solves $Q$.

A decision problem $Q$ is in $NP$ if there exists a non-deterministic polynomial
time algoirthm $A$ that solves $Q$.

\subsection{$P \subseteq NP$}

All problems $P$ are also in $NP$. They simply never use the ``try'' statement
in the pseudocode.

Are there problems that are in $NP$ but not in $P$?
\begin{itemize}
	\item In other words, do we have $P \neq NP$?
	\item Most famouse unsolved problem.
\end{itemize}

\subsection{Reduceability}

A problem $P_x$ reduces to $P_y$ if we can find a mapping function
$f:P_x \to P_y$ such that when $P_y$ answers yes, so does $P_x$ and vice versa.

We say, then that
\[
P_x \leq_P P_y
\]

We know, then, that $P_x$ is no harder than $P_y$, except for some polynomial
factor for the mapping $f$. This translates to
\[
P_y \in P \implies P_x \in P
\]

What about if $P_x \notin P$?
\[
P_x \notin P \implies P_y \notin P
\]




\end{document}
% vim: tw=80
