\documentclass[12pt]{article}
\usepackage{geometry}
\usepackage{amsmath}
\usepackage{amsthm}
\usepackage{amssymb}
\usepackage{mathrsfs}
\usepackage{parskip}
\usepackage{enumerate}
\usepackage{stmaryrd}
\usepackage{listings}
\usepackage{fullpage}
\usepackage{algorithmic}

\begin{document}

\title{CS 341 Notes}
\author{Matthew Visser}
\date{Oct  4, 2011}
\maketitle

\textbf{Readings:}

Closest Pairs --- 33.4\\
Selection --- 8.3 (9.1, 9.2)

\section{Checking Your Guess}

Suppose we guess, with a base case of $n=1$:
\begin{equation}
	T(n) = \begin{cases}
		n \log_3 n + dn & n > 1\\
		1 \log_3 1 = 0 & n=1
	\end{cases}
\end{equation}

Assume true for all $m: 1\leq m < n$.
\begin{equation}
	T^*(m) = m \log_3 m
\end{equation}

Consider $m=n$.
\begin{eqnarray}
	T(n) &=& 3T(\frac{n}{3} + dn\\
	&=& 3 \left( \frac{n}{3} \log_3 \frac{n}{3} \right) + dn \\
	&=&  n \left( \log_3 n - \log_3 3  \right) + dn\\
	&=&  n \log_3n - n + dn\\
	&=& n \log_3 n - n(d-1)
\end{eqnarray}

If we add a constant multiple $k$ then if $k=d$ this cancels out nicely.

Another example\dots

Suppose we guess, with a base case of $n=1$:
\begin{equation}
	T(n) = \begin{cases}
		n \log_3 n  + c& n > 1\\
		c & n=1
	\end{cases}
\end{equation}

Guess $T(m) \leq km$.

$T(1) = c \leq k\cdot 1 = k$

Inductive stype:

Assume $\forall m:1\leq m < n$
\begin{equation}
	T(m) \leq km
\end{equation}

Now consider $T(n)$:
\begin{eqnarray}
	T(n) &\leq& 2T(\frac{n}{2}) + c \\
	&\leq& 2\left(\frac{kn}{2}\right) + c\\
	&=&  kn + c
\end{eqnarray}

Since $c>0$ we can't prove this. But we can change our guess.

Guess $T(m) \leq km - q$

Consider $T(1) = c \leq k\cdot 1 - q$. We can easily pick $k,q$ such that they
satisfy this condition.

Assume $\forall m:1\leq m < n$
\begin{equation}
	T(m) \leq km - q
\end{equation}

Now consider $T(n)$:
\begin{eqnarray}
	T(n) &\leq& 2T(\frac{n}{2}) + c \\
	&\leq& 2\left(\frac{kn}{2}-q\right) + c\\
	&=&  kn -2q + c\\
	&=&  kn - q + (c-q)
\end{eqnarray}

We can assume that $c-q\leq 0$, then $c \leq k\cdot 1 - q$.

Try $k=2c,\ q=c$. This will work out if we prove it with induction.

\section{Closest Pair Problem}

The closest pair problem is to find the two closest points on a plain given a
set of points.

\begin{equation}
	\text{dist}(a,b) = \sqrt{\left( a_1x - b_1x \right)^2 +
	\left( q_1y - b_1y \right)^2}
\end{equation}

\subsection{Brute Force to Something Better}

To brute force this we have $A$, an array of all our points.

\begin{algorithmic}
	\FOR{ $i$ from 1 to $n$}
	\FOR{ $j$ from $i$ to $n$}
	\STATE check dist($A_i,A_j$)
	\IF{ smaller}
	\STATE update min
	\ENDIF
	\ENDFOR
	\ENDFOR
\end{algorithmic}

This is $\Theta(n^2)$. This is terrible, how can we do better? Maybe divide and
conquer?

Maybe we can divide the points after sorting by their $x$ co-ordinate. This is
$\Theta(n \log n)$. After partitioning, we have a left and right side. We can
recursively calculate distances between points on either side.

\begin{algorithmic}
	\STATE $\ell = \text{CP}(\text{Left} A)$
	\STATE $r = \text{CP}(\text{Right} A)$
	\STATE $d = \min(l,r)$
\end{algorithmic}

We can ignore any more checking for points between partitions if the $x$
distance from the right-most point on the left side and left-most point on the
right side is greater than the minimum distance from the left and right side.

We can create a perimiter of the maximum distance we can have between our
dividing point and the points on the left and right side. We will only hav eto
test at mst 6 points (see diagram) to determine if there is a distance that is
less than $d$.


\end{document}

% vim: tw=80
