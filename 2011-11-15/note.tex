\documentclass[12pt]{article}
\usepackage{geometry}
\usepackage{amsmath}
\usepackage{amsthm}
\usepackage{amssymb}
\usepackage{mathrsfs}
\usepackage{parskip}
\usepackage{enumerate}
\usepackage{stmaryrd}
\usepackage{listings}
\usepackage{fullpage}

\begin{document}

\title{CS 341 Notes}
\author{Matthew Visser}
\date{Nov 15, 2011}
\maketitle

\section{Graph Properties}

For a graph $G$ with distinct edge weights\dots

\begin{description}
	\item[Cut Property] Let $S$ be any subset of nodes and let $e$ be the min
		cost edge with exactly one endpoint in $S$. Then any MST contains $e$.
		\begin{proof}
			Let $T$ be an MST without the edge $e$.

			Add $e$ to $T$. We have a cycle in the new graph $T'$. $e$ is in the
			cut-set, and in a cycle, so it's in the intersection of the two as
			well. There is an odd number of more edges that are in the cut set,
			then. One of these edges must already be in $T$, since a spanning
			tree is a connected graph. We can replace $f$ with $e$, and end up
			with another spanning tree. Since $e$ has a weight smaller than $f$,
			this tree has a smaller aggregate weight than $T$, therefore $T$ was
			not a minimal spanning tree, a contradiction.

			Therefore, the cut property holds.
		\end{proof}
	\item[Cycle Property]  Let $C$ be any cycle and let $e$ be the max cost edge
		in $C$. Then every MST does not have edge $e$ in it.
		\begin{proof}
			Let $T$ be an MST with edge $e$ where $e$ has the maximal cost in a
			cycle $C$. Delete $e$ from $T$ to create a cut $S$. Then there is a
			cycle that crosses the cut set, meaning we have two edges that are
			in the cycle and the cut set. Then there is at least one more edge,
			$f$, but by assumption it has a smaller weight than $e$. Then we can
			replace $e$ with $f$ in $T$ for a spanning tree with a smaller
			weight, a contradiction, since $T$ is an MST.

			Therefore the Cycle Property holds.
		\end{proof}
	\item[Cycle-Cut Intersection] A cycle and cut-set must intersect in an even
		number of edges. This is because if a cycle edge crosses the cut, it
		must cross back to complete the cycle.
\end{description}

\section{Other Problems}

\subsection{Travelling Salesman Problem (TSP)}

This is hard, but not provably.

\subsection{Chess Game}

To find out whether or not a colour can win, we must have vertices for every
board state, and edges between them.  Look at the leaf nodes to see if black
wins in one of these.

\section{Algorithm Classification}

We know that polynomial time $\iff$ efficient algorithm. Exponential and
sub-exponential time is bad for algorithm run-time.


\end{document}
% vim: tw=80
